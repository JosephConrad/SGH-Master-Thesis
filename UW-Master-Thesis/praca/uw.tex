\documentclass{pracamgr}  
\usepackage{lmodern} 
\usepackage[polish]{babel} 
\selectlanguage{polish} 
\usepackage{fontspec}
\usepackage{minted}

% package for hyperinks
\usepackage{hyperref}
\hypersetup{
    colorlinks,
    citecolor=black,
    filecolor=black,
    linkcolor=black,
    urlcolor=black
}
   
\definecolor{grey}{gray}{0.9}
\definecolor{bg}{HTML}{FAFAFA}
\definecolor{darkgray}{HTML}{D5D5D5}

\makeatletter
\renewenvironment{minted@colorbg}[1]{
\setlength{\fboxsep}{\z@}
\def\minted@bgcol{#1}
\noindent
\begin{lrbox}{\minted@bgbox}
\begin{minipage}{\linewidth}}
{\end{minipage}
\end{lrbox}%
\colorbox{\minted@bgcol}{\usebox{\minted@bgbox}}}
\makeatother

% Dane magistranta:

\author{Konrad Lisiecki}
\nralbumu{291649}
\title{Mechanizmy funkcyjne w języku C++}
\tytulang{Functional programming in C++}
\kierunek{Informatyka}

\opiekun{dra Marcina Benke\\
  Instytut Informatyki\\
}
\date{Lipiec 2015}
\dziedzina{ 
  11.3 Informatyka\\ 
}
\klasyfikacja{D. Software\\
  D.1 Programming techniques\\
  D.1.1 Applicative (Functional) Programming}
\keywords{C++, functional programming, C++11, funkcja lambda}
\newtheorem{defi}{Definicja}[section]

\begin{document}

\tableofcontents
\maketitle

\begin{abstract}
Celem pracy magisterskiej jest kompleksowe przedstawienie mechanizmów programowania 
funkcyjnego, które zostały wprowadzone do najnowszych standardów języka C++. Autor pracy skupi się
na sposobie realizacji wprowadzonych mechanizmów oraz zbadaniu siły ich wyrazu. 
W ramach pracy zostanie również dokonane ich porównanie z rozwiązaniami istniejącymi w innych 
językach programowania, jak python i haskell.
\end{abstract}


\chapter*{Wprowadzenie}
\addcontentsline{toc}{chapter}{Wprowadzenie}


Język C++ jest jednym z najbardziej popularnych języków programowania.
Powszechnie uważa się, że zastosowano w nim trzy paradygmaty programowania: programowanie imperatywne, obiektowe i generyczne.
Jednak w standardzie C++11 wprowadzono wiele mechanizmów funkcyjnych, znanych z innych języków programowania.

Właśnie o wprowadzonych mechanizmach funkcyjnych do języka C++ traktuje poniższa praca.

Skłąda się ona z~czterech rozdziałów i~dodatków.
W~rozdziale~\ref{r:Zmiany} opisano najważniejsze zmieny jakie wprowadzono do nowego standardu języka C++.
Rozdział~\ref{r:SposobRealizacji} przedstawia sposób wprowadzenia zmian do języka oraz siłę ich wyrazu. 
Wprowadzone zmiany zostaną dokłanie opisane pod względem generyczności i szybkości rozwiązania.


Kolejna część pracy (rozdział \ref{r:Porownanie}) porównuje efektywność i szybkość działania programu.
Zostanie porównana szybkość działania konstrukcji funkcyjnych z tymi spotykanymi w języku Haskell i Python.

Rozdział \ref{r:Biblioteka} zawiera z kolei krótki opis biblioteki zaimplementowanej przy użyciu technik 
metaprogramowania. Jest to biblioteka dostarczająca gotowych funkcji do operawania na listach. Jest ona 
zaimplementowana w duchu biblioteki LINQ, obecnej w języku $C\#$. 


W ostatnim rozdziale jest zawarte podsumowanie wprowadzonych zmian. 
W~dodatkach z kolei, umieszczono wybrane fragmenty kodu zaimplementowanej biblioteki z rozdziału \ref{r:Biblioteka}.


\chapter{Opis wprowadzonych zmian do języka C++}\label{r:Zmiany}

Pojęciem pierwotnym blabalii fetorycznej jest \emph{blaba}.
Blabaliści nie podają jego definicji, mówiąc za Ciach-Pfe t-\=am
K\^un (fooistyczny mędrzec, XIX w. p.n.e.):
\begin{quote}
  Blaba, który jest blaba, nie jest prawdziwym blaba.

\raggedleft\slshape tłum. z~chińskiego Robert Blarbarucki
\end{quote}


std::find_if(container.begin(), container.end(),
             [](int val) { return 0 < val && val < 10; });


\section{Funkcje lambda}

Funkcje lambda, chociaż nie zwiększają siły wyrazu języka C++, moją olbrzymi wpływ na jakość pisanego 
oprogramowania. Przede wszystki
 
\begin{listing}[ht]
\inputminted[mathescape, linenos, numbersep=5pt, bgcolor=bg, rulecolor=\color{darkgray}, frame=lines, framesep=2mm]{cpp}
{listings/lambda0.cpp}
\caption{Przykład funkcji wyższego rzędu w języku Python}
\label{lst:HigherOrderPython}
\end{listing}



\begin{minted}[mathescape, linenos, numbersep=5pt, bgcolor=bg, gobble=2, rulecolor=\color{darkgray}, frame=lines, framesep=2mm]{cpp}
  string title = "This"
  /*
  Defined as $\pi=\lim_{n\to\infty}\frac{P_n}{d}$ where $P$ is the perimeter
  of an $n$-sided regular polygon circumscribing a
  circle of diameter $d$.
  */
  const double pi = 3.1415926535

\end{minted}

\begin{listing}[ht]
\inputminted[mathescape, linenos, numbersep=5pt, gobble=2, frame=lines, framesep=2mm]{cpp}{lambda1.cpp}
\caption{Przykład poprawnego i niepoprawnego użycia capture-list w wyrażeniach lambda}
\label{listing:3}
\end{listing}


\begin{listing}[ht]
\inputminted[mathescape, linenos, numbersep=5pt, frame=lines, framesep=2mm]{cpp}{firstCategory.cpp}
\caption{Przykład poprawnego i niepoprawnego użycia capture-list w wyrażeniach lambda}
\label{listing:3}
\end{listing}

\begin{listing}[ht]
\inputminted[mathescape, linenos, numbersep=5pt, gobble=2, frame=lines, framesep=2mm]{cpp}{accumulate.cpp}
\caption{Przykład poprawnego i niepoprawnego użycia capture-list w wyrażeniach lambda}
\label{listing:3}
\end{listing}



\section{Automatyczne określanie typu}
\section{Funkcje obywatelami pierwszej kategorii}
\section{Funkcje wyższego rzędu}

Wprowadzaenie do standardu C++11 funkcji anonimowych znaczanie ułatwiło korzystanie z STL-owych funkcji, takich jak:
\begin{enumerate}
\item $trasform$
\item $remove\_if$
\item $accumulate$
\end{enumerate}

Obecnie nie ma potrzeby definiowania nowej funkcji z nazwy, można po prostu użyć funkcji lambda.
Efektem jest znaczące uproszczenie kodu, co czyni go bardziej czytelnym i napisanym w bardziej funkcyjnym stylu.
 

\begin{defi}\label{skupienie}
  Silny, zwarty i gotowy fetor bazowy nazwiemy \emph{skupieniem}.
\end{defi}

\begin{defi}\label{fetor}
  \emph{Fetorem} nazwiemy skupienie blaba spełniające następujący
  \emph{aksjomat reperkusatywności}:
  $$\forall \mathcal{X}\in Z(t)\ \exists
  \pi\subseteq\oint_{\Omega^2}\kappa\leftrightarrow 42$$
\end{defi}
 
\begin{figure}[tp]
  \centering
  \framebox{\vbox to 4cm{\vfil\hbox to
      7cm{\hfil\tiny.\hfil}\vfil}}
  \caption{Artystyczna wizja blaba w~obrazie węgierskiego artysty
    Josipa~A. Rozkoszy pt.~,,Blaba''}
\end{figure}









\chapter{Zbadanie sposobu realizacji oraz ich siły wyrazu}\label{r:SposobRealizacji}
 








\chapter{Porównanie z innymi językami}\label{r:Porownanie}

W nieniejszym rozdziale zostanie przedstawione porównanie jak wygląda szybkość funkcji 
wyższego rzędu w C++11 (z zastosowaniem funkcji anonimowych), w odniesieniud do innych języków.
Do celów porównawczych wybrano język Haskell, który jest językiem funkcyjnym bez efektów ubocznych.
Dokonano również porównania w stosounku do Pythona, który ma zaimplementowane wiele 
funkcji wyższego rzędu w swojej bibliotece standardowej.

\section{Wydajność funkcji wyższego rzędu w C++11}


Na listingu nr \ref{lst:HigherOrderCPP} możemy zobaczyć przykład implementacji 3 funkcji wyżeszgo rzędu 
z użycie funkcji lambda. 

\begin{listing}[ht]
\inputminted[mathescape, linenos, numbersep=5pt, frame=lines, framesep=2mm]{cpp}{../SEM1/fun.cpp}
\caption{Przykład poprawnego i niepoprawnego użycia capture-list w wyrażeniach lambda}
\label{lst:HigherOrderCPP}
\end{listing}



\section{Porównanie z językiem Haskell}
\section{Porównanie z językiem Python}

W tej sekcji zostanie przedstawione te same funkcje wyższego rzędu, tylko że w języku Pyhthon. 
Znajdują się one na listingu nr \ref{lst:HigherOrderPython}:
 
\begin{listing}[ht]
\inputminted[mathescape, linenos, numbersep=5pt, frame=lines, framesep=2mm]{python}{../SEM1/fun.py}
\caption{Przykład funkcji wyższego rzędu w języku Python}
\label{lst:HigherOrderPython}
\end{listing}









\chapter{Biblioteka LINQ w C++}\label{r:Biblioteka}

Głównym odkryciem Fifaka jest, że fetor suprakowariantny może
gryzmolizować dowolny ideał w~podprzestrzeni widłowej przestrzeni
lamblialnej funkcji Rozkoszy.

Udowodnienie tego faktu wymagało wykorzystania twierdzeń pochodzących
z~kilku niezależnych teorii matematycznych (zob. na przykład:
\cite{russell,spyrpt,JR,beaman,hopp,srinis}).  Jednym z~filarów
dowodu jest teoria odwzorowań owalnych Leukocyta (zob.~\cite{leuk}).

Znaczenie twierdzenia Fifaka dla problemu blabalizy polega na tym, że
znając retroizotonalne współczynniki dla klatek Rozkoszy można
przeprowadzić fetory bazowe na dwie nieskończone bazy fetorów $\sigma$
w~przestrzeni $K_7$ i~fetorów $\rho$ w~odpowiedniej
quasi-quasi-przestrzeni równoległej (zob.~\cite{hopp}).  Zasadnicza
różnica w~stosunku do innych metod blabalizy polega na tym, że
przedstawienie to jest dokładne.


\chapter{Podsumowanie}

W~pracy przedstawiono pierwszą efektywną implementację blabalizatora
różnicowego.  Umiejętność wykonania blabalizy numerycznej dla danych
,,z życia'' stanowi dla blabalii fetorycznej podobny przełom, jak dla
innych dziedzin wiedzy stanowiło ogłoszenie teorii Mikołaja Kopernika
i~Gryzybór Głombaskiego.  Z~pewnością w~rozpocznynającym się XXI wieku
będziemy obserwować rozkwit blabalii fetorycznej.

Trudno przewidzieć wszystkie nowe możliwości, ale te co bardziej
oczywiste można wskazać już teraz.  Są to:
\begin{itemize}
\item degryzmolizacja wieńców telecentrycznych,
\item realizacja zimnej reakcji lambliarnej,
\item loty celulityczne,
\item dokładne obliczenie wieku Wszechświata.
\end{itemize}

\section{Perspektywy wykorzystania w~przemyśle}

Ze względu na znaczenie strategiczne wyników pracy ten punkt uległ
utajnieniu.

\appendix

\chapter{Główna pętla programu zapisana w~języku T\=oFoo}

\begin{verbatim}
[[foo]{,}[[a3,(([(,),{[[]]}]),
  [1; [{,13},[[[11],11],231]]].
  [13;[!xz]].
  [42;[{,x},[[2],{'a'},14]]].
  [br;[XQ*10]].
 ), 2q, for, [1,]2, [..].[7]{x}],[(((,[[1{{123,},},;.112]],
        else 42;
   . 'b'.. '9', [[13141],{13414}], 11),
 [1; [[134,sigma],22]].
 [2; [[rho,-],11]].
 )[14].
 ), {1234}],]. [map [cc], 1, 22]. [rho x 1]. {22; [22]},
       dd.
 [11; sigma].
        ss.4.c.q.42.b.ll.ls.chmod.aux.rm.foo;
 [112.34; rho];
        001110101010101010101010101010101111101001@
 [22%f4].
 cq. rep. else 7;
 ]. hlt
\end{verbatim}

\chapter{Przykładowe dane wejściowe algorytmu}

\begin{center}
  \begin{tabular}{rrr}
    $\alpha$ & $\beta$ & $\gamma_7$ \\
    901384 & 13784 & 1341\\
    68746546 & 13498& 09165\\
    918324719& 1789 & 1310 \\
    9089 & 91032874& 1873 \\
    1 & 9187 & 19032874193 \\
    90143 & 01938 & 0193284 \\
    309132 & $-1349$ & $-149089088$ \\
    0202122 & 1234132 & 918324098 \\
    11234 & $-109234$ & 1934 \\
  \end{tabular}
\end{center}

\chapter{Przykładowe wyniki blabalizy
    (ze~współczynnikami~$\sigma$-$\rho$)}

\begin{center}
  \begin{tabular}{lrrrr}
    & Współczynniki \\
    & Głombaskiego & $\rho$ & $\sigma$ & $\sigma$-$\rho$\\
    $\gamma_{0}$ & 1,331 & 2,01 & 13,42 & 0,01 \\
    $\gamma_{1}$ & 1,331 & 113,01 & 13,42 & 0,01 \\
    $\gamma_{2}$ & 1,332 & 0,01 & 13,42 & 0,01 \\
    $\gamma_{3}$ & 1,331 & 51,01 & 13,42 & 0,01 \\
    $\gamma_{4}$ & 1,332 & 3165,01 & 13,42 & 0,01 \\
    $\gamma_{5}$ & 1,331 & 1,01 & 13,42 & 0,01 \\
    $\gamma_{6}$ & 1,330 & 0,01 & 13,42 & 0,01 \\
    $\gamma_{7}$ & 1,331 & 16435,01 & 13,42 & 0,01 \\
    $\gamma_{8}$ & 1,332 & 865336,01 & 13,42 & 0,01 \\
    $\gamma_{9}$ & 1,331 & 34,01 & 13,42 & 0,01 \\
    $\gamma_{10}$ & 1,332 & 7891432,01 & 13,42 & 0,01 \\
    $\gamma_{11}$ & 1,331 & 8913,01 & 13,42 & 0,01 \\
    $\gamma_{12}$ & 1,331 & 13,01 & 13,42 & 0,01 \\
    $\gamma_{13}$ & 1,334 & 789,01 & 13,42 & 0,01 \\
    $\gamma_{14}$ & 1,331 & 4897453,01 & 13,42 & 0,01 \\
    $\gamma_{15}$ & 1,329 & 783591,01 & 13,42 & 0,01 \\
  \end{tabular}
\end{center}

\listoffigures
\listoftables

\begin{thebibliography}{99}
\addcontentsline{toc}{chapter}{Bibliografia}

\bibitem[Bea65]{beaman} Juliusz Beaman, \textit{Morbidity of the Jolly
    function}, Mathematica Absurdica, 117 (1965) 338--9.

\bibitem[Blar16]{eb1} Elizjusz Blarbarucki, \textit{O pewnych
    aspektach pewnych aspektów}, Astrolog Polski, Zeszyt 16, Warszawa
  1916.

\bibitem[Fif00]{ffgg} Filigran Fifak, Gizbert Gryzogrzechotalski,
  \textit{O blabalii fetorycznej}, Materiały Konferencji Euroblabal
  2000.

\bibitem[Fif01]{ff-sr} Filigran Fifak, \textit{O fetorach
    $\sigma$-$\rho$}, Acta Fetorica, 2001.

\bibitem[Głomb04]{grglo} Gryzybór Głombaski, \textit{Parazytonikacja
    blabiczna fetorów --- nowa teoria wszystkiego}, Warszawa 1904.

\bibitem[Hopp96]{hopp} Claude Hopper, \textit{On some $\Pi$-hedral
    surfaces in quasi-quasi space}, Omnius University Press, 1996.

\bibitem[Leuk00]{leuk} Lechoslav Leukocyt, \textit{Oval mappings ab ovo},
  Materiały Białostockiej Konferencji Hodowców Drobiu, 2000.

\bibitem[Rozk93]{JR} Josip A.~Rozkosza, \textit{O pewnych własnościach
    pewnych funkcji}, Północnopomorski Dziennik Matematyczny 63491
  (1993).

\bibitem[Spy59]{spyrpt} Mrowclaw Spyrpt, \textit{A matrix is a matrix
    is a matrix}, Mat. Zburp., 91 (1959) 28--35.

\bibitem[Sri64]{srinis} Rajagopalachari Sriniswamiramanathan,
  \textit{Some expansions on the Flausgloten Theorem on locally
    congested lutches}, J. Math.  Soc., North Bombay, 13 (1964) 72--6.

\bibitem[Whi25]{russell} Alfred N. Whitehead, Bertrand Russell,
  \textit{Principia Mathematica}, Cambridge University Press, 1925.

\bibitem[Zen69]{heu} Zenon Zenon, \textit{Użyteczne heurystyki
    w~blabalizie}, Młody Technik, nr~11, 1969.


\end{thebibliography}

\end{document}


%%% Local Variables:
%%% mode: latex
%%% TeX-master: t
%%% coding: latin-2
%%% End:
