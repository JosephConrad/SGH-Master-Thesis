\documentclass{pracamgr}  
\usepackage{lmodern} 
\usepackage[polish]{babel} 
\selectlanguage{polish} 
\usepackage{fontspec}
\usepackage{minted}
\usepackage{listings}
% package for hyperinks
\usepackage{hyperref}
\usepackage{svg}
\hypersetup{
    colorlinks,
    citecolor=black,
    filecolor=black,
    linkcolor=black,
    urlcolor=black
}
\usepackage{graphicx}  
\usepackage{makeidx}\makeindex
\linespread{1.5} 
 
\definecolor{grey}{gray}{0.9}
\definecolor{bg}{HTML}{FAFAFA}
\definecolor{darkgray}{HTML}{D5D5D5} 

\makeatletter
\renewenvironment{minted@colorbg}[1]{
\linespread{1.0} 
\setlength{\fboxsep}{\z@}
\def\minted@bgcol{#1}
\noindent
\begin{lrbox}{\minted@bgbox}
\begin{minipage}{\linewidth}}
{\end{minipage}
\end{lrbox}%
\colorbox{\minted@bgcol}{\usebox{\minted@bgbox}}}
\makeatother

%Jesli uzywasz kodowania polskich znakow ISO-8859-2 nastepna linia powinna byc odkomentowana
%Jesli uzywasz kodowania polskich znakow CP-1250 to ta linia powinna byc 
%odkomentowana
%\usepackage[cp1250]{inputenc}

% Dane magistranta:

\author{Konrad Lisiecki}
\nralbumu{48211}
\title{Wycena Opcji przy użyciu Modeli Zmienności Stochastycznej} 
\kierunek{Finanse i rachunkowość}
\instytut{Ekonometrii}
\opiekun{dra hab. Łukasza Delonga} 
\date{Warszawa 2015}   

% Tu jest dobre miejsce na Twoje własne makra i~środowiska:
\newtheorem{defi}{Definicja}[section]

% koniec definicji 

%\makeindex[intoc] 
 


%===========================================================================
%             Bibliogrphy
%=========================================================================== 
\usepackage[style=numeric,sorting=ydnt,defernumbers=true, backend=bibtex]{biblatex}
\addbibresource{biblio.bib}
 

%===========================================================================
%                           Begin of document 
%===========================================================================


\begin{document}
\maketitle
\nocite{book-full} 

%===========================================================================
%                               Introduction
%===========================================================================
\chapter*{Streszczenie} 


Tematem pracy magisterskiej jest wycena opcji przy pomocy modeli zmienności stochastycznej. 
W standardowych modelach służących do wyceny opcji przyjmuje się (jak np. w modelu 
Blacka-Scholesa), że zmienność jest wartością stałą, niezależną od czasu. 
Jednak jak dalekie jest to założenie od rzeczywistości pokazał chociażby ostatni kryzys 
ekonomiczny, gdzie kursy instrumentów finansowych wahały się o wiele bardziej niż w czasach
koniunktury gospodarczej. Dlatego też, w modelach służących do wyceny opcji uzmiennia się parametr opisujący zmienność instrumentów finansowych i uzależnia się go od czasu. 


Po wstępie, w drugim rozdziale pierwszym zostanie przedstawiony schemat budowania symulacji w oparciu o metodę Monte Carlo. Będzie się to odbywało na przykładzie wyceny opcji w modelu Blacka-Scholesa, który 
jest jednym z najprostszych modeli do wyceny opcji.

W trzecim rozdziale zostanie opisany model Hestona i sposób w jaki można wyliczyć. 
Zostanie przedstawione kroki jakie należy poczynić, aby wycenić opcję przy pomocy tego modelu.

Czwarty rozdział z kolei jest poświęcony kalibracji modelu, czyli inaczej ujmując, znalezieniu 
optymalnej wartości wektora nieznanych parametrów modelu. Optymalna wartość jest w tym przypadku zdefiniowana
jako taka, dla której wartość opcji wyliczonej na podstawie modelu Hestona jest najbliższa wartości 
obserowowanej na rynku.

W przedostatnim rozdziale z kolei zostaną przedstawione możliwości dalszych rozszerzeń modelu Hestona. 
Polegają one głównie na próbie uzmiennieniu (uzależnienia od czasu) tych wartości parametrów, które dla 
modelu Hestona są stałe w czasie.

Ostatni rozdział to już próba empirycznego zbadania zachowania modelu Hestona. Zostanie w nim porównana
wartość opcji na indeks \textit{S\&P} stosując odpowiednio model Blacka-Scholesa i model Hestona.



%===========================================================================
%                               Table of contents
%===========================================================================
\tableofcontents
 

% \addcontentsline{toc}{chapter}{Introduction} \markboth{INTRODUCTION}{}


%===========================================================================
%                               Wprowadzenie
%===========================================================================
\chapter{Wprowadzenie}
\label{chap:introduction}


\begin{quote}
We developed what is known a stochastic volatility model. This is a model where the volatility as well as the underlying asset price moves around in an unpredictable way.
\raggedleft\slshape John Hull \index{Hull John}
\end{quote}
Niniejsza praca jest poświęcona modelom zmienności stochastycznej używanych przy wycenie opcji. 
Jednak za tą dosyć enigmatyczną nazwą, nasuwa się pytanie, czym dokładnie są te modele? 
Jaka jest ich istota?

Bardzo zwięzłej, a zarazem trafnej odpowiedzi na te pytania udziela John Hull, 
jeden z pionierów dzisiejszych finansów ilościowych. Stwierdza on, zgodnie z przytoczonym 
powyżej cytatem, że są to modele, gdzie nie tylko cena aktywa bazowego, ale również jego zmienność, 
poruszają się w sposób nieprzewidywalny, losowy. 

Jednak do lat '90 XX wieku podstawowym modelem do wyceny opcji był Model Blacka-Scholesa. U jego podstaw 
leży bardzo istotne założenie mówiące o tym, że zmienność instrumentów finansowych jest stała w czasie. 

Niestety, założenie to jest niezgodne z tym, co możemy obserwować na rynkach finansowych. Można to szczególnie mocno zauważyć w czasie cyklicznie powtarzających się kryzysów finansowych.

Stąd też, wysoka, niestała zmienność na rynkach finansowych stała się podstawową motywacją wprowadzenia modeli wyceny
opcji dla których zmienność nie jest ustalonym parametrem, a zmienną zależną od pewnych innych czynników. 
Pionierem w tej obszarze okazał się Steven Heston, który w 1993 opublikował pracę, gdzie wprowadza
nowy model wyceny opcji, w którym nie tylko proces cen instrumentu bazowego, ale również
proces zmienności jest stochastyczny (losowy) i zależny od czasu \cite{Heston}. Na jego cześć, model ten 
w literaturze znany jest pod nazwą modelu Hestona.

Nieniejsza praca stanowi próbę przedstawienia tego modelu oraz sposobu jak, używając istniejących narzędzi,
można przy jego pomocy wyliczyć wartość opcji. Część emiryczna pracy jest z kolei próbą sprawdzenia, jak 
model, oraz narzędzia do jego stosowania, sprawdzają się w praktyce.


\chapter{Metoda Monte Carlo}
\label{chap:introduction}

\begin{quote}

Never think that lack of variability is stability. Don't confuse lack of volatility with stability, ever.
 
\raggedleft\slshape Nassim Nicholas Taleb \index{Taleb Nassim Nicholas}
\end{quote}
 

Rozdział ten stanowi wprowadzenie do tematyki poruszonej w pracy, której jest wycena opcji w oparciu
o modele stochastycznej zmienności. 
Jednak zanim przejdziemy do tego przejdziemy przedstawiony zostanie ogólny schemat postępowania, na przykładzie modelu mniej skomplikowanego, \textbf{modelu Blacka-Scholesa}, w którym zmienność jest stała w czasie.
Jako przykład wyceniona zostanie najprostsza opcji \textit{call}, a narzędziem do tego służącym 
będzie \textbf{metoda Monte Carlo}.

Niezależnie od rodzaju opcji, czy też modelu opisującego dynamikę cen aktywa bazowego, można 
wyróżnić takie same etapy symulacji. 

Zaliczamy do nich:
\begin{enumerate}
  \item Generowanie zmiennych losowych o ustalonym rozkładzie
  \item Wyznaczenie wartości zmiennych losowych
  \item Kalibracja modelu
  \item Wyznaczenie wartości opcji
  \item Symulacja Monte Carlo
\end{enumerate}

W niniejszym rozdziale zostanie przedstawiony po krótce każdy z tych elementów wraz z odpowiednim przykładem opisującym kolejny krok wyceny opcji, zgodnie z modelem Blacka-Scholesa.
Wycena na podstawie modelu Hestona będzie podobna do tej przedstawionej w niniejszym rozdziale.

\section{Generowanie zmiennych losowych o ustalonym rozkładzie}

Pierwszym krokiem jaki należy podjąć podczas przeprowadzania symulacji Monte Carlo jest 
wygenerowanie zmiennych losowych o ustalonym rozkładzie.

I tak, w przypadku wyceny opcji w Modelu Blacka-Scholesa potrzebna jest umiejętność generowania liczb (pseudo)losowych z rozkładu normalnego. Znamy wiele algorytmów generowania liczb o rozkładzie normalnym, wśród których można wyróżnić:
\begin{enumerate}
  \item metoda odwrotnej dystrybuanty
  \item meteda Boxa-Mullera
  \item metoda biegunów (\textit{ang. Marsaglia Polar Method})
  \item algorytm ziggurat
\end{enumerate}

Mimo że, najbadziej wydajną metodą generowania liczb pseudolosowych o rozkładzie normalnym jest algorytm ziggurat, w tym rozdziale zostanie przedstawiona metoda polarna.

Jest ona wariantem metody Boxa-Mullera, jednak jest od niego znacznie wydajniejsza.
Polega ona na wybraniu losowego punktu spełniającego warunek:
\begin{equation}
  s = x^2 + y^2 < 1
\end{equation}
Wtedy, aby otrzymać liczby losowe o standardowym rozkładzie normalnym, należy zastosować następujące przekształcenia:
\begin{equation}
  u_1 = x \sqrt{\frac{-2ln(s)}{s}}, u_2 = y \sqrt{\frac{-2ln(s)}{s}}
\end{equation}
Poniższy kod przedstawia implementację tego algorytmu:

\begin{listing}[H]
\inputminted[mathescape, linenos, numbersep=5pt, bgcolor=bg, frame=lines, framesep=2mm]{cpp}
{../src/heston/SimpleMonteCarlo/PolarGenerator.cpp}
\caption{Generowanie zmiennych o rozkładzie normalnym zgodnie z metodą biegunów}
\label{lst:lambdaSyntax}
\end{listing} 

\section{Wyznaczenie wartości zmiennych losowych}

Obliczenie ceny aktywa bazowego w momencie wygaśnięcia opcji jest najważeniejszym elementem 
wyceny opcji. 

Aby wycenić opcję rozważmy następujący model przedstawiający ewolucję ceny aktywa
bazowego w czasie oraz obligacji, jako instrumentu finansowego pozbawionego ryzyka:
\begin{equation}
  dS_t = \mu S_t dt + \sigma S_t d W_t 
\end{equation}
\begin{equation}
  dB_t = r B_t dt, B_0 = 1
\end{equation}
gdzie:
zatem $B_t = E^{rt}$

Model ten znany jest pod nazwą \textbf{modelu Blacka-Scholesa} i jest fundamentalnym narzędziem do wyceny opcji.
Zgodnie z teorią wyceny opcji wypłata $f$, w momencie wygaśnięcia opcji $T$ jest równa:

\begin{equation}
  e^{-rT} E(f(S_T))
\end{equation}
gdzie dynamikę cen aktywa bazowego opisuje następujący proces w mierze obojętnej na ryzyko (mierze martyngałowej):

\begin{equation}
  dS_t = r S_t dt + \sigma S_t d W_t 
\end{equation}


\section{Kalibracja modelu}
Kalibracja jest jednym z najważniejszych elementów podczas wyceny opcji przy pomocy jakiegokolwiek
modelu.  

W przypadku modelu Blacka Scholesa, jedynym niobserwowalnym parametrem zmienność $\sigma$ (współczynnik dyfuzji).
Można go wyznaczyć na jeden z dwóch sposobów:
\begin{enumerate}
  \item zmienność historyczna
  \item zmienność implikowana
\end{enumerate}

Szczególnie interesującym jest zmienność implikowana, ponieważ wyraża ona \textit{"przewidywania rynku"} co 
do wartości zmienności danego aktywa bazowego. 
Można ja w prosty sposób obliczyć przy użyciu analitycznego \textbf{wzoru Blacka-Scholesa} na wycenę opcji, który 
możemy przedstawić jako:

\begin{equation}
  C_{obs} = C(S_0, T, K, \sigma_{imp}, r)
\end{equation}
Jak widać, wiedząc jaka jest cena opcji na podstawie danych rynkowych, równanie staje się równaniem z jedną niewiadomą. 

Jednakże, w przypadku bardziej skomplikowanych modeli, nieznanych parametrów może być dużo więcej. Wtedy też 
stosuje się bardziej zaawansowane techniki kalibracyjne, co będzie tematyką rozdziału \ref{chap:chapterModelCalibration}.
  % \autoref - ok but it uses the english name of chapter


\section{Wyznaczenie wartości opcji}

Będąc uzbrojonym w wartości wszystkich parametrów modelu Blacka-Scholesa, można przystąpić
do wyliczenia wartości opcji. I tak, jak wiadomo, wartość opcji europejskiej dla strony kupującej 
wynosi:

\begin{equation}
  V = max(S-K, 0)
\end{equation}

W powyższym wzorze $K$, czyli cena wykupu opcji jest parametrem, stąd też,  aby wyliczyć wartość opcji należy obliczyć wartość instrumentu bazowego w momencie wygaśnięcia opcji.

W modelu tym, wartość tą możemy wyznaczyć przy pomocy następującego wzoru analitycznego:
\begin{equation}
  haha TODO
\end{equation}

Aby uzyskać cenę opcji na dany moment, należy zdyskontować wartość opcji obliczonej w momencie wykupu opcji.

\section{Symulacja Monte Carlo}

Ostatnim krokiem, który pozostał jest wyznaczenie prawdziwej wartości opcji, co możemy uzyskać 
stosując symulację metodą Monte Carlo.

W tej pracy zostanie użyta aproksymacja metodą Monte Carlo. Metoda Monte Carlo jest sposobem na wyznaczenie 
Poniżej jest przedstawiony formalny 
Jeżeli $X_1, ..., X_n ~P$ iid  wtedy
jest średnia z próby 
\begin{equation}
  
\end{equation}
Wtedy $\mu$ jest podstawowym estymatorem Monte Carlo $Ef(x)$, gdzie $X$ (X ma również rozkład P). Inaczej mówiąc, aproksymujemy prawdziwą wartość wartością oczekiwaną \textbf{średnią z próby} \textit{ang. sample mean}.

Implementację metody Monte-Carlo, wraz z przedstawionym w tym rozdziale sposobem wyznaczenia 
ceny opcji w oparciu o model Blacka-Scholesa, przedstawia listing nr \ref{lst:MCBS}:




\begin{listing}[H]
\inputminted[mathescape, linenos, numbersep=5pt, bgcolor=bg, frame=lines, framesep=2mm]{cpp}
{../src/heston/SimpleMonteCarlo/MCBlackScholes.cpp}
\caption{Symulacja Monte Carlo wyliczająca cenę opcji kupna (zgodnie z modelem Blacka-Scholesa) o zadanym zestawie paramterów}
\label{lst:MCBS}
\end{listing} 


 




%===========================================================================
%
%                               Model Hestona
%
%===========================================================================
\chapter{Model Hestona}
\label{chap:hestonModel}
\begin{quote}

  Suppose we use the standard deviation ... of possible future returns on
  a stock ... as a measure of its volatility. Is it reasonable to take
  that volatility as a constant over time? I think not.

\raggedleft\slshape Fisher Black \index{Fisher, Black}
\end{quote}


Obecnie jednym z najpopularniejszych narzędzi do wyceny opcji jest model Blacka-Scholsa. Ceniony jest on ze względu na prostotę oraz wygodę użycia, kosztem jednak wielu upraszczających założeń. 


\section{Motywacja} 
Jak zostało powiedziane we wstępie do niniejszego rozdziału, podstawowym modelem do wyceny opcji jest model Blacka-Scholesa.
Umożliwia on wyprowadzenie wzoru na cenę opcji europejskiej w postaci analitycznej. Jednakże, podczas gdy jego prostota jest jedną z jego największych zalet, to posiada on wiele założeń, które nie przystają do rzeczywistości rynkowej. Pierwsza grupa to założenia dotyczące aktywów:
\begin{enumerate}
\item cena aktywów bazowych ma rozkład lognormalny
\item zmienność aktywów bazowych i jest stała i znana z góry
\item akcje nie wypłacają dywidend
\item wyceniane opcje są opcjami Europejskimi, tzn. przy pomocy modelu Blacka-Scholesa nie można wycenić np. opcji Amerykańskich
\end{enumerate}
Z kolei druga grupa założeń odnosi się do rynku na którym dane aktywa występują.
\begin{enumerate}
\item na rynku nie ma możliwości osiągnięcia ponadnormatywnego zysku bez ryzyka (brak możliwości arbitrażu)
\item nie ma kosztów transakcyjnych.
\item stopa procentowa na rynku $r_t$ jest stała i znana z góry 
\end{enumerate}

Wszystkie te założenia sprawiają, że model Blacka-Scholesa, mimo, że pozwala na wyznaczenie ceny opcji w postaci analitycznej, może nieprecyzyjnie wycenić wartość takiej opcji.
Jednym z takich założeń jest to o stałej zmienności instrumentu bazowego. Świadczy o tym chociażby
wykres nr \ref{fig:vix}, na którym wyraźnie widać wysoką zmienność indeksu S\&P. Jednym z pomysłów na obejście tego problemu jest uzmiennienie tej stałej, czyli dopuszczenie, aby dla dowolnego czasu, parametr ten przyjmował różną wartość. Można to zrobić np. poprzez nadanie zmienności cech losowych. W takim przypadku nie tylko proces ceny akcji byłby procesem stochastycznym, ale także aby sama zmienność byłaby definiowana przy użyciu procesu stochastycznego. Takie też podejście zostało wykorzystane przy budowie modelu Hestona. \cite{greenwade93}



\begin{figure}
  \centering
  \includegraphics[width=0.80\textwidth]{corr.png}
  \caption{Zmieność w czasie indeksu S\&P}\label{fig:vix}
\end{figure} 

\section{Model Hestona}
Jak wspomniano w poprzednim rozdziale, model Hestona eliminuje podstawową wadę modelu Blacka-Scholesa jakim jest założenie o stałej zmienności w czasie.
W tym celu zmienność również uzależniono od procesu losowego, tworząc kolejny proces stochastyczny. W przypadku modelu Hestona jest to model Coxa-Ingersolla-Rossa (model CIR):
\begin{equation}
dr_t  = \kappa (\theta  - r_t)dt + \epsilon \sqrt{r_t} dW_t^v 
\end{equation}

Model CIR jest często wykorzystywany przy modelowaniu zmienności stopy procentowej (stąd też oznaczenie $r_t$) ze względu na \textbf{\textit{własność powrotu do średniej długoterminowej}}. 
(\textit{ang. mean-reversion}). Gdy zamiast $r_t$ podstawimy $v_t$ otrzymamy proces przedstawiający zmienność zmienności w czasie. Założenie to jest także prawdziwe w przypadku rynku 
akcji. Jeżeli byśmy założyli, że zmienność nie ma własności powrotu do średniej, obserwowalibyśmy znaczną część aktywów z bardzo gwałtownie rosnącą zmiennością lub będącą blisko zeru.
Ponadto, tak jak w przypadku stopy procentowej, tak w przypadku akcji występuje własność powrotu do średniej, gdy cena akcji nie spadnie poniżej 0 \cite{TestingMeanReversion}.
Co więcej, gdy proces spełnia warunek Fellera:
\begin{equation}
2 \kappa \theta > \epsilon^2
\end{equation}
proces jest ściśle dodatni \cite{TheLittleHestonTrap}.


Co więcej zmienność jest zawsze jest wartością większą od zera, co wynika wprost z jej definicji. Stąd też w drugiej części wzoru mamy funkcję pierwiastkową z $v_t$: 

\begin{equation} 
dv_t  = \kappa (\theta - v_t)dt + \epsilon \sqrt{v_t} dW_t^v 
\end{equation}

Mając równanie opisujące proces zmienności można już przedstawić model Hestona, który ma postać:

\begin{equation}
dS_t  = \mu S_t dt + \sqrt{v_t} S_t dW^S_t
\end{equation}
gdzie wariancja $v_t$ jest opisana przez następujące równanie: 

\begin{equation}
dv_t  = \kappa (\theta - v_t)dt + \epsilon \sqrt{v_t} dW_t^v 
\end{equation}
Procesy $dW_t^v$ oraz $dW_s^v$ są standardowymi procesami Wienera i są skorelowane:

\begin{equation}
Cov[dW^S_t, dW^v_t] = \rho dt 
\end{equation}
gdzie:

\begin{enumerate}
\item $\mu$ oznacza dryft (\textit{ang. drift}). ceny aktywa bazowego 
\item $\theta$ oznacza długoterminową wartością oczekiwaną $v_t$
\item $\kappa$ jest współczynnikiem szybkości powrotu do średniej (\textit{ang. rate of mean-reversion})
\item $\epsilon$ oznacza zmienność zmienności, czyli wariancję $v_t$
\end{enumerate}

W dodatku, w przedstawiony modelu występuje równanie $Cov[dW^S_t, dW^v_t] = \rho dt $. Mówi ono o tym, że 
dwa procesy błądzenia losowego są w rzeczywistości ze sobą skorelowane przy pomocy stałego współczynnika 
korelacji $\rho$.
To założenie jest zgodne z rzeczywistością. Często możemy na rynkach finansowych zaobserwować następującą sekwencję zdarzeń: gdy cena rośnie gwałtownie w górę, zwiększa się zmienność, 
natomiast w okresach spokoju zmienność jest na relatywnie niskim poziomie.

Niestety, w odróżnieniu od modela Blacka-Scholsa, dla modelu Hestona nie istnieje pełne rozwiązanie analitycznie (istnieje tylko częściowe rozwiązanie analityczne). Wobec tego, aby należy się posłużyć numerycznymi metodami wyznacznia rozwiązania i przeprowadzimy symulacje Monte Carlo, aby wyznaczyć 
cenę opcji zgodną z tym modelem.


% Poniższa lista przedstawia podstawowe 
% \begin{enumerate}
% \item rozkład logarytmów stóp zwrotu nie jest normalny
% \item zmienność jest zmienna w czasie
% \item autokorelacja kolejnych stóp zwrotu
% \item trajektorie mają skoki
% \end{enumerate}

% W kolejnych rozdziała 


% %===========================================================================
% %                               Model Hestona
% %===========================================================================
% \section{Model Hestona}

% Na początek załóżmy, że cena aktywa bazowego w momencie $t$ spełnia następujący proces dyfuzji:
% \begin{equation}
% dS(t)= \mu S dt + \sqrt{v(t)} S d z_1 (t),
% \end{equation}
% gdzie $z_1(t)$  jest standardowym procesem Wienera. 

% Jeżeli natomiast zmienność jest procesem Ornsteina-Uhlenbecka:

% \begin{equation}
% d \sqrt{v(t)} = - \beta \sqrt{v(t)} dt + \delta d z_2 (t)
% \end{equation}

% wtedy na podstawie lematu Ito można wykazać, że wariancja $v(t)$ ma następującą postać:
% \begin{equation}
% dv(t)= \kappa  [\theta -v(t)]dt+2\delta \sqrt{v(t)} d z_2 (t)
% \end{equation}
% gdzie proces $z_2(t)$ ma korelację z procesem $z_1(t)$ równą $\rho$.

% Gdy przyjmiemy stałą stopę procentową, cena w momencie $t$ obligacji, która wygasa w chwili $t+\tau$ wynosi:
% \begin{equation}
% P(t, t+\tau) = e^{-r\tau}
% \end{equation}



% \begin{figure}
%   \centering
%   \includegraphics[width=150mm]{vix.jpg}
%   \caption{Zmieność w czasie indeksu S\&P}\label{fig:vix}
% \end{figure}


% gdzie parametr $\sigma$ jest stały. Założenie to jednak okazuję się być zbyt upraszczające dla szerokiej gamy szeregów czasowych. Spójrzmy np. na szereg czasowy  (\ref{fig:vix}) 



% \begin{equation}\label{h:ito}
% dv(t) = [\delta^2 - 2 \beta v(t)] dt + 2\beta \sqrt{v(t)} dz_2(t),
% \end{equation}

% The (\ref{h:ito}) equation can be rewritten to the following, square root process

% \begin{equation}\label{h:squareroot}
% dv(t) = \kappa [\theta - v(t)] dt + \sigma \sqrt{v(t)} d_2 (t),
% \end{equation}


\section{Dyskretyzacja Eulera}

Jednak, przeprowadzenie wyceny opcji w oparciu o model Hestona jest zadaniem znacznie trudniejszym
niż to było chociażby w przypadku modelu BS. Wynika to z tego, że w modelu Hestona trajektoria 
ceny instrumentu bazowego jest teraz zależna od trajektorii jego zmienności. To pociąga za sobą 
duże konsekwencje w procesie wyceny opcji, ponieważ teraz należy zasymulować najpierw trejktorię
zmienności, a dopiero później, na tej podstawie, generować trajektorię cen. 

Stąd też, aby wygenerować trajektorię cen, musimy być w stanie przedstawić równanie opisującą
zmienność aktywa bazowego w postaci \textit{zdyskretyzowanej}.

Poniższe równanie nr \ref{h:euler} przedstawia \textbf{dyskretyzację Eulera} omawianego równania:
 
\begin{equation}\label{h:euler}
v_{i+1}  = v_i + \kappa (\theta - v_i) \Delta t + \epsilon +  \sqrt{v_i} \Delta W^{v}_{i+1}
\end{equation}

Jednakże, teraz jesteśmy w miejscu, gdzie dyskretyzujemy proces ciągły do procesu dyskretnego. 
W takim przypadku może pojawić się, ze względu na błąd dyskretyzacji, ujamna wartość procesu $v_{i+1}$.

Aby tego uniknąć, zgodnie z literaturą (przypis TODO), wszędzie tam, gdzie możliwa jest ujemna 
wartość zmienności $v_i$, zastępujemy ją poprzez $v_i^+ = \max(v_i, 0)$. Ostatecznie więc, wzór
opisujący dynamikę $v_i$ przyjmuje postać:

\begin{equation}\label{h:eulerNonZero}
v_{i+1}  = v_i + \kappa (\theta - v_i^+) \Delta t + \epsilon +  \sqrt{v_i^+} \Delta W^{v}_{i+1}
\end{equation}

Mając wygenerowaną trajektorię zmienności, można przystąpić do zdyskretyzowania równania opisującego 
dynamikę cen aktywa bazowego jednakże pamiętając o tym, aby zmienność zawsze była wartością dodatnią. 
Mając to na uwadzę, równanie nr \ref{h:eulerAssetPath} opisuję dynamikę cen instrumentu bazowego w postaci dyskretnej:

\begin{equation}\label{h:eulerAssetPath}
S_{i+1} = S_i e^{\mu - \frac{1}{2} v_i^+} \Delta t + \sqrt{v_i^+ \Delta t}  \Delta W_{i+1}^S
\end{equation}
  
\section{Generowanie skorelowanych szeregów czasowych}

Kolejnym kwestią, która pojawia się w implementacji modelu Hestona, jest generowanie skorelowanych 
szeregów czasowych (wygenerowanie zmiennych losowych $dW^S_t$ oraz $dW^v_t$). 


O ile jednak dla przypadku wielowymiarowego, generowanie skorelowanych szeregów czasowych jest dosyć
skomplikowane i czasochłonne obliczeniowo, to dla przypadku dwuwymiarowego problem redukuje się do 
następującej postaci:
\begin{equation}
  \epsilon_1 = x_1
  \epsilon_2 = \rho x_1 + x_2 \sqrt{1-\rho^2}
\end{equation}

Poniższy wykres nr \ref{} TODO przedstawia przykład skorolowanych szeregów czasowych o zadanym
stopni korelacji $\rho = 0.2$:




\section{Symulacja Monte Carlo}


Na poniższym listing nr \ref{lst:MCHeston} znajduje się główna funkcja użyta do wygenerowania ceny 
opcji w oparciu o model Hestona:

\begin{listing}[H]
\inputminted[mathescape, linenos, numbersep=5pt, bgcolor=bg, frame=lines, framesep=2mm]{cpp}
{../src/heston/MonteCarlo/MCHeston.cpp}
\caption{Generowanie zmiennych o rozkładzie normalnym zgodnie z algorytmem Boxa-Mullera }
\label{lst:MCHeston}
\end{listing} 




%===========================================================================
%                       Kalibracja modelu Hestona
%===========================================================================
\chapter{Kalibracja modelu}
\label{chap:chapterModelCalibration}


Mając wyznaczony dla którego chcielibyśmy wyznaczyć cenę opcji, wciąż potrzebujemy parametrów dla tego modelu. 
Problem wyznaczenia parametrów modelu nazywamy \textbf{problemem kalibracji} modelu.  

Jednym ze sposobem na kalibrację modeli jest technika symulowanego wyżarzania (ang. simulated annealing). Należy od do grona stochastycznych metaheurystyk, które pomagają w rozwiązaniu problemów optymalizacyjnych (a takim właśnie problemem jestkalibracja modelu, chcemy zminimalizować wartość funkcji będącej różnicą wartości funkcji wynikającej z modelu i z rynku).

Kolejnym sposobem jest powszechnie stosowana nielinowa metoda najmniejszych kwadratów. I właśnie ta metoda zostanie użyta
w niejszej pracy w celu wyznaczenia parametrów modelu.

Aby jednak efektywnie zastosować powyższe metody, należy znaleść równie efektywny sposób wyznaczania wartości opcji w modelu Hestona. Ponieważ podejście symulacyjne jest zbyt czasochłonne w procesie kalibracji, w tym przypadku stosuje się pół zamknięte równanie opisuące cenę opcji.

\section{Symulowane wyżarzanie}

Jednym ze sposobów skalibrowania modelu stochastycznego jest algorytm symulowanego wyżarzania  (\textit{ang. simulated annealing}) oraz. Rozwiązuje on szereg problemów podczas rozwiązywania problemów optymalizacyjnych, a najważniejszym z nich jest potencjalne istnienie wielu ekstremów lokalnych.
W tym przypadku większość algorytmów, które są monotoniczne, nigdy nie osiągnie ekstremum globalnego. Jedną z takich technik jest \textit{hill climbing}, która osiąga dobre rezultaty tylko dla funkcji o wypukłej powierzchni. Aby znaleść rozwiązanie tego problemu odchodzi się od zasady monotoniczności, tzn. algorytm nie zawsze wybiera rozwiązanie bardziej optymalne od poprzedniego. 
Wśród możliwych podejść należy wymienić np.:
\begin{enumerate}
  \item pełny przegląd badanego obszaru
  \item podejście deterministyczne w algorytmie \textit{tabu search}
  \item podejście randomizacyjne (\textit{algorytm Metropolisa, symulowane wyżarzanie} \cite{OptimalizationBySimulatedAnnealing} )
\end{enumerate}


W nieniejeszej pracy podejście, jakim starano się skalibrować model Hestona, jest podejście randomizacyjne.
Schemat algorytmu symulowanego wyżarzania można zobaczyć na poniższym listingu:

\begin{listing}[H]
\inputminted[mathescape, linenos, numbersep=5pt, bgcolor=bg, frame=lines, framesep=2mm]{cpp}
{listings/sa.cpp}
\caption{Algorytm symulowanego wyżarzania }
\label{lst:lambdaSyntax}
\end{listing}


\section{Nieliniowa Metoda Najmniejszych Kwadratów}

Tematem tego podrozdziału jest zastosowany w nienjszej pracy oraz zaimplementowany w \textit{Matlabie} funkcji \textit{lsqnonlin}. Funkcja ta szuka najbardziej optymalnego rozwiązania przy pomocy metody \textit{NMNK} (nieliniowej metody najmniejszych kwadratów \textit{ang. nonlinear least squares}) minimalizując funkcję celu:

\begin{equation}
  min_x \norm{f(x)}^2_2 = min_x(f_1(x)^2 + f_2(x)^2 + \cdots + f_n(x)^2)
\end{equation}

Postać funkcji celu, niezbędnej do obliczenia żądanych parametrów modelu Hestona, jest tematach kolejnego podrozdziału.

\section{Pół-zamknięte równanie opisujące cenę opcji w modelu Hestona}

Jak wspomniano we wstępie, model Hestona ma częściowe rozwiązanie zamknięte. Można je zdefiniować, analogicznie do modelu Blacka-Scholesa, jako:

\begin{equation}
  C(S, v, t) = SP_1 -K P(t,T) P_2
\end{equation}

gdzie:

\begin{equation}
  x = ln[S]
\end{equation}

\begin{equation}
  P_j (x, v, T; ln[K]) = \frac{1}{2} + \frac{1}{\pi} \int_{0}^{\infty} Re \bigg[ \frac{e^{-i \cdot \phi ln[K]} f_j(x, v, T; \phi) }{i \phi} d \phi \bigg]
\end{equation}

Gdzie funkcje charakterystyczne, $\phi_1, \phi_2$ mają postać: 

\begin{equation}
   f_j(x, v, T; \phi) = e^{C(T-t; \phi) + D(T-t; \phi)v + i \phi x}
\end{equation}

\begin{equation}
  C (\tau; \phi) = r \phi i \tau + \frac{a}{\sigma^2} \bigg\{ (b_j) - \rho \sigma \phi i + d) \tau - 2 ln (\frac{1 - ge^{dr}}{1-g} \bigg\}
\end{equation}

\begin{equation}
D (r; \phi)  = \frac{b_j- \rho \sigma \phi i + d}{\sigma^2} \bigg[ \frac{1 - e^{d\tau}}{1 - ge^{d\tau}} \bigg]  
\end{equation}

\begin{equation}
g= \frac{b_j - \rho \sigma \phi i + d}{b_j - \rho \sigma \phi i - d}
\end{equation}

Poniższy listing przedstawia implementację powyższego wzoru obliczającego cenę opcji w modelu Hestona:

\begin{listing}[H]
\inputminted[mathescape, linenos, numbersep=5pt, bgcolor=bg, frame=lines, framesep=2mm]{matlab}
{../src/HestonCalibration/HestonCalibration/HestonCall.m}
\caption{Obliczanie ceny opcji z półzamkniętego wzoru na model Hestona}
\label{lst:lambdaSyntax}
\end{listing}


\begin{listing}[H]
\inputminted[mathescape, linenos, numbersep=5pt, bgcolor=bg, frame=lines, framesep=2mm]{matlab}
{../src/HestonCalibration/HestonCalibration/hestoncalibrationexample.m}
\caption{Obliczanie ceny opcji z półzamkniętego wzoru na model Hestona}
\label{lst:lambdaSyntax}
\end{listing}


Aby obliczyć parametry modelu jako dane wejściowe podajemy następujące dane dostępne na rynku:
\begin{enumerate}
  \item cenę wykonania opcji (ang. strike)
  \item czas wygaśnięcia opcji (ang. maturity)
  \item zmienność implikowaną (ang. implied volatility)
  \item timetillexpiry
\end{enumerate}
Jako następujące zostały w

%===========================================================================
%                       Rozszerzenia modelu Hestona
%===========================================================================
\chapter{Rozszerzenia modelu Hestona}

Model Hestona, jak opisano w poprzednim rozdziale, odrzuca założenie o stałości zmienności w czasie. 
Jednak pozostałe parametry wciąż pozostają na nie zmienionym poziomie, co daje możliwość uzmiennienia
kolejnych stały w modelu. Jest to jedna z możliwych zmian w modelu Hestona. 

Druga możliwość to wprowadzenie do modeli zmienności stochastycznej skoków (\textit{ang. stochastic volatility models with jumps})
Tego typu modele tworzą odrębną klasę modeli, wsród których można wyróżnić model Hestona ze skokami, 
model Bates czy też model Barndorff-Nielsena i Shepharda.

\section{Modele uzmiennniające stałe w modelu Hestona} % (fold)
\label{sec:modele_uzmiennniajace}


\section{Modele zmienności stochastycznej ze skokami} % (fold)
\label{sec:modele_zmienno_ci_stochastycznej_ze_skokami}

% section modele_zmienno_ci_stochastycznej_ze_skokami (end)
% section modele_uzmiennniaj_cd (end)




% %===========================================================================
% %                       Model Hestona na przykładznie
% %===========================================================================
% \chapter{Optymalne decyzje inwestycyjne}\label{r:sp}
% Jak wiadomo zmienność nie jest bezpośrednio obserwowalna na rynkach finansowych.
% Aby wnioskować o zmienności, można:
% \begin{enumerate}
% \item wykorzystać technikę filtrowania (ang. filtering)
% \item ceny opcji z rynku lub indeksy zmienności rynkowej (ang. market volatility indices)
% \end{enumerate}

% Ustalająć $\mu =\frac{1}{2}$ 
% Gdzie parametr $\mu$ oznacza wrażliwość współczynnika dyfuzji względem poziomu zmienności.


% \begin{enumerate}
% \item Rozszerzony model Hestona z procesem CEV (ang. constant elasticity variance):

% Model CEV wyraża się przy pomocy następującego wzoru:
% \begin{equation}
% dS = \mu S dt  + \sigma_0 S^{\frac{}{}}
% \end{equation}

% Zmienność jest więc w tym przypadku funkcją ceny akcji (stochastycznej).


% \begin{equation}

% dV_t = k_V(\hat{V}-V_t)dt+ \hat{V}^{\frac{1}{2}-\mu} V_t^{\mu} (b_{SV} dB_t^S+ h_VdB_t^V)
% \end{equation}
% \item ceny opcji z rynku lub indeksy zmienności rynkowej (ang. market volatility indices)

% \item Rozszerzony model Stein-Steina (ang. extended Stein-Stein model):

% Równanie:

% \begin{equation}
% \frac{dS_t}{S_t} = (R_t + \lambda_t \sigma_t) dt + \sigma_t + d B_t^S
% \end{equation}

% \begin{equation}
% d\lambda_t  = \kappa_{\lambda} (\hat{\lambda} - \sigma_t) dt 
% + \beta_{\lambda S} dB_t^S + g_{\sigma}dB_t^{\sigma}
% \end{equation}


% \begin{equation}
% d\sigma_t  = \kappa_{\sigma} (\hat{\sigma} - \sigma_t) dt 
% + \beta_{\sigma S} dB_t^S + g_{\sigma}dB_t^{\sigma} + g_{\lambda}
% dV_t^{\lambda}
% \end{equation}

% gdzie:
% $(B_t^S, B_t^{\sigma}, B_t^{\lambda})$ - są ortogonalnymi, 
% wielowymiarowymi procesami Wienera.

% \item Model Hestona 

% \begin{equation}
% \frac{}{} = (R_t + lV_t)dt + \sqrt{V_t} dB^S_t
% \end{equation}

% Proces opisujący wariancję $V_t$ ma nastepujący proces pierwiastowy:



% \end{enumerate}

%===========================================================================
%                       Wycena opcji walutowych
%===========================================================================
% \chapter{Wycena opcji walutowych}\label{r:sp}

% \section{Model Blacka-Scholesa}

% \section{Model Garmana-Kohlhagena}

% Model Garmana-Kohlagena jest zaadaptowanym na potrzebu rynku walutowego modelem
% Blacka-Scholesa o znanej stopie dywidendy . 


%===========================================================================
%                       Model Hestona na przykładznie
%===========================================================================
\chapter{Wycena opcji na indeks S\&P}\label{r:sp}

W tym rozdziale zostanie przedstawione zastosowanie opisanej w poprzednich 
rozdziałach teorii do wyceny opcji na indeks S\&P. 

\section{Wycena opcji przy pomocy modlu Blacka-Scholesa}

Zanim przejdziemy do wyceny opcji  przy pomocy modelu Hestona oraz kalibracji
parametrów tego modelu, warto, w celach porównawczych zbadać wartość opcji 
wyznczonej przy pomocy modelu Blacka-Scholesa. 

\begin{figure}
  \centering
  \includegraphics[width=0.80\textwidth]{../figures/blackScholesVolSurface.pdf}
  \caption{Płaszczyzna zmienności implikowanej dla indeksu $S\&P$}
  \label{fig:volatilitySurface}
\end{figure}


\section{Kalibracja}

W tym rozdziale skalibrujemy parametry modelu Hestona, tak aby ona ich 
podstawie model z najmniejszym błędem wyznaczał wartość opcji.

Rysunek \ref{fig:data} przedstawia pokazuje zrzutowane na wykresie dane, które
zostały użyte do skalibrowania modelu.

\begin{figure}
  \centering
  \includegraphics[width=0.80\textwidth]{../MySavedPlot.png}
  \caption{Zmieność w czasie indeksu S\&P}
  \label{fig:data}
\end{figure}

\begin{figure}

Jak widać z powyższego rysunku, do skalibrowania modelu zostały użyte ceny 
opcji o 3 różnych terminach wygaśnięcia: 6, 13 oraz 20 dni.


\begin{itemize}
  \item $V(1) = 0.0534 $
  \item $\kappa = 17.1316$
  \item $\theta = 0.0000 $
  \item $\sigma = 0.5000$
  \item $\rho = -0.9000 $
\end{itemize}

 
Na poniższym rysunku widać wynik symulacji ceny opcji dla 10000 tysięcy przebiegów.
Z definicji Monte Carlo wynika, ze poprawnym estymatorem dla wartości oczekiwanej ceny
opcji jest średnia ze wszystkich przebiegów, zatem na rysunku przedstawiono także linię
oznaczającą średnią wartość ceny opcji. 


\centering
  \includegraphics[width=0.80\textwidth]{../chartHeston.png}
  \caption{Zmieność w czasie indeksu S\&P}
  \label{fig:data}
\end{figure}


Z wykomania symulacji dla metody Hestona wynika ze srednia cena opcji z czesem do wygasniecia jednego roku wunisi 

Warronrowniez wykonac porownanie ceny opcji z C na wyznaczona ptzy pomocy mod lu Blacka Scholesa. Opcja o takich samych parametrach zostala wyloczona na podstawie metody przedstawionej w rozdziale pierwszym. Cena takiej opcji wynosi , takze jest bardzo bliska wartosci wyznacznej przy pomocy mod lu Blacka Scholesa.

W tym miejscunwarro rowniez porownac wyznaczona cene z tym jak rynek. Na rynku opcja o bardzo podobnych parametrach (jest wyceniana na ). C na ta nieznacznie rozni od tych wyznaczonych przy pomocy metod wyznaczonej w tej pracy, jednak nalezy wziac pod uwage fakry, z nawet male roznice w przyjetych parametrach moga powodowac roznicw c enia. Np. Stope na wolna od ryzyka w niniejszej pracy oszacowano na poziomi 3 $\%$, zgodnie z rentownoscia 30-letnich obligacji wystawiony przez rzad Stanow Zjednoczonych. Niewykluczone, ze rentownosc dla ktorej cena opcji na rynku przyjmuj taka warosc jest nieznacznie inna,

\section{Analiza wrażliwości}

Jednym z ważniejszych parametrów, które można wyznaczyć podczas procesu opcji, są współczynniki
wrażliwości rynku opcji od przyjętych paramtrów. Są to tzw. \textbf{współczynniki greckie}.
Zaliczamy do nich:
\begin{enumerate}
  \item $\Delta$ (delta)
  \item $\Gamma$ (gamma)
  \item $\Theta$ (theta)
  \item $v$ (vega)
  \item $\rho$ (rho)
\end{enumerate}

W pracy przedstawimy zachowanie się modelu Hestona dla jednego, wybranego wspóczynnika greckiego, 
a mianowicie dla $\Theta$.

\textit{Theta}, jako współczynnik cen opcji w zależności od upływającego czasu:
\begin{equation}
  \Theta = - \frac{\delta V}{\delta T}
\end{equation}

Rysunek \ref{fig:hestonTimeToExpiry} przedstawia zachowanie się cen opcji wraz z rosnącym czasem do 
wygaśnięcia opcji. Dla pierwszego wykresu czas do wygaśnięcia wynosi pół roku, natomiast dla ostatniego 
wynosi 6 lat. Wyraźnie widać, że trajektorie ceny instrumentu bazowego o wiele bardziej odchylają się 
na 
\begin{figure}
  \includegraphics[width=1.00\textwidth]{../figures/hestonTimeToExpiry.pdf}
  \caption{Cena instrumentu bazowego w modelu Hestona w zależności od czasu do wygaśnięcia opcji}
  \label{fig:hestonTimeToExpiry}
\end{figure}

Jak można zauważyć, wraz z spadającym czasem do wygaśnięcia maleje potencjalny rozrzut cen aktywa bazowego w momencie wygaśnięcia opcji
Jest to zgodne z intuicją. Opcje, których cena jest zależna od zmienności instrumentu bazowego, 
w krótszym okresie czasu będą będą mniej warte od tych z dłuższym okresem do wygaśnięcia.


%===========================================================================
%                             Zakończenie
%===========================================================================
 \chapter*{Zakończenie}\label{r:ending}

Celem niejszej pracy było porównanie wydajności modeli służących do wyceny
opcji oraz implementacja modelu Hestona, który wylicza wartość opcji biorąc
pod uwagę brak stałości w czasie zmienoności, jednego z podstawowych paramatrów
mającego wpływ na wycenę opcji. 

Pierwszym modelem służacy do wyceny opcji jest model Blacka-Scholesa, który jest
najprostszym modelem do wyceny opcji i porównano go do efektywności modelu 
Hestona. 

Model Hestona, który jest bardziej zaawansowany koncepcyjnie, jest jednoscześnie 
bardziej skomplikowany obliczeniowo. Wynika to z faktu, że ma on kilka dodatkowych
parametrów, które należy wyznaczyć na podstawie informacji z rynku. Proces wyznaczania tych 
parametrów, nazwany procesem kalibracji, jest o wiele bardziej skomplikowany w przypadku 
modelu Hestona, ponieważ tutaj jest to zadania optymalizacyjne, natomiast w przypadku modelu
Blacka-Scholesa jest zadania algebraiczne.


Z badań emirycznych jasno wynika, że TODO


Podsumowując, można stwierdzić, że model Hestona jest o wiele bardziej zaawansowanym
narzędziem do wyceny opcji niż model Blacka-Scholesa. Uwzględniając zróżnicowanie zmienności 
w czasie sprawia, że cena opcji jest bardziej dokładna. Jednak cena jaką za to płacimy 
jest o wiele większe skomplikowanie modelu, zarówno w sferze koncepcyjnej, jak i 
implementacyjnej. Ze względu na konieczność wieloparametrowej kalibrcji modelu, 
model ten jest o wiele cięższy obliczeniwo, niż chociażby model Blacka-Scholesa.


\appendix

% \chapter{Basic concepts and definitions}

% \begin{verbatim}
% [[foo]{,}[[a3,(([(,),{[[]]}]),
%   [1; [{,13},[[[11],11],231]]].
%   [13;[!xz]].
%   [42;[{,x},[[2],{'a'},14]]].
%   [br;[XQ*10]].
%  ), 2q, for, [1,]2, [..].[7]{x}],[(((,[[1{{123,},},;.112]],
%         else 42;
%    . 'b'.. '9', [[13141],{13414}], 11),
%  [1; [[134,sigma],22]].
%  [2; [[rho,-],11]].
%  )[14].
%  ), {1234}],]. [map [cc], 1, 22]. [rho x 1]. {22; [22]},
%        dd.
%  [11; sigma].
%         ss.4.c.q.42.b.ll.ls.chmod.aux.rm.foo;
%  [112.34; rho];
%         001110101010101010101010101010101111101001@
%  [22%f4].
%  cq. rep. else 7;
%  ]. hlt
% \end{verbatim}
 
 


% \begin{center}
%   \begin{tabular}{rrr}
%     $\alpha$ & $\beta$ & $\gamma_7$ \\
%     901384 & 13784 & 1341\\
%     68746546 & 13498& 09165\\
%     918324719& 1789 & 1310 \\
%     9089 & 91032874& 1873 \\
%     1 & 9187 & 19032874193 \\
%     90143 & 01938 & 0193284 \\
%     309132 & $-1349$ & $-149089088$ \\
%     0202122 & 1234132 & 918324098 \\
%     11234 & $-109234$ & 1934 \\
%   \end{tabular}
% \end{center}

% %\chapter{Exemplary results}

% \begin{center}
%   \begin{tabular}{lrrrr}
%     & Coefficients \\
%     & haha & $\rho$ & $\sigma$ & $\sigma$-$\rho$\\
%     $\gamma_{0}$ & 1,331 & 2,01 & 13,42 & 0,01 \\
%     $\gamma_{1}$ & 1,331 & 113,01 & 13,42 & 0,01 \\
%     $\gamma_{2}$ & 1,332 & 0,01 & 13,42 & 0,01 \\
%     $\gamma_{3}$ & 1,331 & 51,01 & 13,42 & 0,01 \\
%     $\gamma_{4}$ & 1,332 & 3165,01 & 13,42 & 0,01 \\
%     $\gamma_{5}$ & 1,331 & 1,01 & 13,42 & 0,01 \\
%     $\gamma_{6}$ & 1,330 & 0,01 & 13,42 & 0,01 \\
%     $\gamma_{7}$ & 1,331 & 16435,01 & 13,42 & 0,01 \\
%     $\gamma_{8}$ & 1,332 & 865336,01 & 13,42 & 0,01 \\
%     $\gamma_{9}$ & 1,331 & 34,01 & 13,42 & 0,01 \\
%     $\gamma_{10}$ & 1,332 & 7891432,01 & 13,42 & 0,01 \\
%     $\gamma_{11}$ & 1,331 & 8913,01 & 13,42 & 0,01 \\
%     $\gamma_{12}$ & 1,331 & 13,01 & 13,42 & 0,01 \\
%     $\gamma_{13}$ & 1,334 & 789,01 & 13,42 & 0,01 \\
%     $\gamma_{14}$ & 1,331 & 4897453,01 & 13,42 & 0,01 \\
%     $\gamma_{15}$ & 1,329 & 783591,01 & 13,42 & 0,01 \\
%   \end{tabular}
% \end{center}

\listoffigures
\addcontentsline{toc}{chapter}{Lista rysunków} \markboth{Figures}{}
\listoflistings
\addcontentsline{toc}{chapter}{Lista kodów źródłowych} \markboth{Listings}{}



% \chapter*{Bibliografia}
% \addcontentsline{toc}{chapter}{Bibliografia} \markboth{Bibliography}{}

% \begin{quote}

% If I have seen farther than others, it is because I was standing on the shoulders of giants.

% \raggedleft\slshape Isaac Newton \index{Isaac Newton}
% \end{quote}


\printbibliography
 


\printindex
\end{document}


%%% Local Variables:
%%% mode: latex
%%% TeX-master: t
%%% coding: latin-2
%%% End: